%%%%%%%%%%%%%%%%%%%%%%%%%%%%%%%%%%%%%%%%%
% Professional Formal Letter
% LaTeX Template
% Version 2.0 (12/2/17)
%
% This template originates from:
% http://www.LaTeXTemplates.com
%
% Authors:
% Brian Moses
% Vel (vel@LaTeXTemplates.com)
%
% License:
% CC BY-NC-SA 3.0 (http://creativecommons.org/licenses/by-nc-sa/3.0/)
%
%%%%%%%%%%%%%%%%%%%%%%%%%%%%%%%%%%%%%%%%%

%----------------------------------------------------------------------------------------
%	PACKAGES AND OTHER DOCUMENT CONFIGURATIONS
%----------------------------------------------------------------------------------------

\documentclass[11pt, a4paper]{letter} % Set the font size (10pt, 11pt and 12pt) and paper size (letterpaper, a4paper, etc)

\input{structure.tex} % Include the file that specifies the document structure

%\longindentation=0pt % Un-commenting this line will push the closing "Sincerely," and date to the left of the page

%----------------------------------------------------------------------------------------
%	YOUR INFORMATION
%----------------------------------------------------------------------------------------


\Who{Luiz Max de Carvalho} % Your name

\Title{, PhD} % Your title, leave blank for no title
\authordetails{
	School of Applied Mathematics\\ % Your department/institution
	Praia de Botafogo, 190\\ % Your address
	Rio de Janeiro, RJ, 22250-900\\ % Your city, zip code, country, etc
	Email: lmax.fgv@gmail.com \\ % Your email address
	Phone: +55 21 3799-2348 \\ % Your phone number
% 	URL: LaTeXTemplates.com % Your URL
}

%----------------------------------------------------------------------------------------
%	HEADER CONTENTS
%----------------------------------------------------------------------------------------
\logo{emap.png}
% \logo{Marca_FGV_EMAp_colorida.png}

\headerlinetwo{Getúlio Vargas Foundation (FGV)} % Top header line, leave blank if you only want the bottom line

% \headerlinetwo{School of Applied Mathematics (EMAp)} % Bottom header line

%----------------------------------------------------------------------------------------

\begin{document}

%----------------------------------------------------------------------------------------
%	TO ADDRESS
%----------------------------------------------------------------------------------------

\begin{letter}{
	Prof. Jane-Ling Wang\\
	Department of Statistics\\
	University of California\\
	Davis, CA 95616, USA
}

%----------------------------------------------------------------------------------------
%	LETTER CONTENT
%----------------------------------------------------------------------------------------

\opening{Dear Professor Wang,}

I would like to submit the attached manuscript for consideration for publication in~\textit{JASA}.
The paper is being submitted for the first time.
It extends the approach of Poole \& Raftery (2000)\footnote{Poole D, Raftery AE. Inference for deterministic simulation models: the Bayesian melding approach. Journal of the American Statistical Association. 2000 Dec 1;95(452):1244-55.} to accommodate varying weights in a logarithmic pool (LP).
Instead of focusing on Bayesian melding only, however, we offer a more general study of log-linear mixtures and their statistical applications.
A number of caveats are also transparently discussed.

In the same fashion as other~\textit{JASA} papers such as~Gneiting \& Raftery (2007)\footnote{Gneiting T, Raftery AE. Strictly proper scoring rules, prediction, and estimation. Journal of the American statistical Association. 2007 Mar 1;102(477):359-78.}, we offer a mix of review of important results and new ones.
We offer a concise but inclusive exposition of the main properties of logarithmic pooling, as well as some new results such as log-pooling for exponential families, unique log-concavity of LP and non-uniqueness of the entropy-optimised weights.
We then bring the main innovation of the paper, which are the hierarchical priors on the weights, discuss prior choice in this setting and explore issues of identifiability and interpretation in detail.
To complete exposition, we provide a wealth of applications, from simple probability estimation to meta-analysis to Bayesian melding for deterministic models -- including epidemic models.
The paper also leaves open many new avenues of research, such as using logarithmic pooling for Bayesian model averaging and eliciting priors for the weights from external information.

This would be of interest to the readership of~\textit{JASA} because it sheds light on an overlooked part of the statistical literature while offering motivation for the use of logarithmic pooling (i.e, its mathematical properties) and  a clear exposition of the potential applications in a broad class of problems.
The paper is the culmination of over five years of work.
We are indebted to the people listed in the Acknowledgments section of the paper, who helped us solidify ideas and hone many of the arguments/experiments in the paper over the years. 
This is some of my best work and I would be glad if you would consider it for publication.

\closing{Cordially,}

%----------------------------------------------------------------------------------------
%	OPTIONAL FOOTNOTE
%----------------------------------------------------------------------------------------

% Uncomment the 4 lines below to print a footnote with custom text
%\def\thefootnote{}
%\def\footnoterule{\hrule}
%\footnotetext{\hspace*{\fill}{\footnotesize\em Footnote text}}
%\def\thefootnote{\arabic{footnote}}

%----------------------------------------------------------------------------------------

\end{letter}

\end{document}
