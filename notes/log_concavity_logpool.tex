\documentclass[a4paper, notitlepage, 10pt]{article}
\usepackage{geometry}
\fontfamily{times}
\geometry{verbose,tmargin=30mm,bmargin=25mm,lmargin=25mm,rmargin=25mm}
% \pagestyle{empty}
% end configs
\usepackage{setspace,relsize}               
\usepackage{moreverb}                        
\usepackage{url}
\usepackage{hyperref}
\hypersetup{colorlinks=true,citecolor=blue}
\usepackage{amsmath}
\usepackage{mathtools} 
\usepackage{amsthm}
\usepackage{amssymb}
\usepackage{indentfirst}
\usepackage{todonotes}
\usepackage[authoryear,round]{natbib}
\bibliographystyle{apalike}
\usepackage[pdftex]{lscape}
\usepackage[utf8]{inputenc}

% Title Page
\title{\vspace{-9ex}\centering \bf Logarithmic pooling and log-concavity}
\author{
Luiz Max F. de Carvalho\\
% Program for Scientific Computing (PROCC), Oswaldo Cruz Foundation. \\
% Institute of Evolutionary Biology, University of Edinburgh.\\
}
\DeclareMathOperator*{\argmin}{arg\,min}
\DeclareMathOperator*{\argmax}{arg\,max}
\newtheorem{theorem}{Theorem}[]
\newtheorem{proposition}{Proposition}[]
\newtheorem{remark}{Remark}[]
\newtheorem{definition}{Definition}[]
\newtheorem{lemma}{Lemma}[]
\setcounter{theorem}{0} % assign desired value to theorem counter
\begin{document}
\maketitle

\begin{abstract}
In this brief note I claim to show that logarithmic pooling is the~\textit{only} pooling operator that will~\textit{always} produce a log-concave opinion when all expert opinions are also log-concave.

Key-words: logarithmic pooling; log-concavity; uniqueness. 
\end{abstract}

\section*{Background}

Logarithmic pooling is a popular method for combining opinions on an agreed quantity, specially when these opinions can be framed as probability distributions.
Let $\mathbf{F_\theta} := \{f_0(\theta), f_1(\theta), \ldots, f_K(\theta)\}$ be a set of distributions representing the opinions of $K + 1$ experts and let $\boldsymbol\alpha :=\{\alpha_0, \alpha_1, \ldots, \alpha_K \} \in \mathcal{S}^K$ be the vector of weights, such that $\alpha_i > 0\: \forall i$ and $\sum_{i=0}^K \alpha_i = 1$, i.e., $\mathcal{S}^{K + 1}$ is the space of all open simplices of dimension $K + 1$.
The \textbf{logarithmic pooling operator} $\mathcal{LP}(\mathbf{F_\theta}, \boldsymbol\alpha)$ is defined as
\begin{equation}
\label{eq:logpool}
 \mathcal{LP}(\mathbf{F_\theta}, \boldsymbol\alpha) :=  \pi(\theta | \boldsymbol\alpha) = t(\boldsymbol\alpha) \prod_{i=0}^K f_i(\theta)^{\alpha_i},
\end{equation}
where $t(\boldsymbol\alpha) = \int_{\boldsymbol\Theta}\prod_{i=0}^K f_i(\theta)^{\alpha_i}d\theta$.
This pooling method enjoys several desirable properties and yields tractable distributions for a large class of distribution families~\citep{genest1984,genest1986A}.

Another desirable property of the logarithmic pooling operator is log-concavity.
Log-concavity of the pooled prior may be important to consider in order to guarantee unimodality and certain conditions on tail behaviour~\citep{Bagnoli2005}.

\begin{definition}
\label{def:RPC}
\textbf{Relative propensity consistency~\citep{genest1984}}.
Taking $\boldsymbol F_{X}$ as a set of expert opinions with support on a space $\mathcal{X}$, define $\boldsymbol \xi = \{\boldsymbol F_{X}, a, b\}$ for arbitrary $a , b \in \mathcal{X}$.
Let $\mathcal{T}$ be a pooling operator and define two functions $U$ and $V$ such that 
\begin{align}
 U(\boldsymbol \xi) &:= \left( \frac{f_0(a)}{f_0(b)}, \frac{f_1(a)}{f_1(b)}, \ldots, \frac{f_K(a)}{f_K(b)} \right)\:\text{and}\\
 V(\boldsymbol \xi) & := \frac{\mathcal{T}_{\boldsymbol F_{X}} (a)}{\mathcal{T}_{\boldsymbol F_{X}} (b)}.
\end{align}
We then say that $\mathcal{T}$ enjoys \textit{relative propensity consistency} (RPC) if and only if
\begin{equation}
 U(\boldsymbol \xi_1) \geq U(\boldsymbol \xi_2) \implies  V(\boldsymbol \xi_1) \geq V(\boldsymbol \xi_2),
\end{equation}
for all $\boldsymbol \xi_1, \boldsymbol \xi_2$.
\end{definition}
Informally, this property says that if all experts consider a particular event $A$ more probable than another event $B$, then the pooled opinion should be consistent with these relative judgments. 

\begin{lemma}
\label{lem:RPC_LP}
\textbf{Uniqueness of LP for RPC }~\citep{genest1984}.
Logarithmic pooling is the~\textbf{only} pooling operator that enjoys RPC.
\end{lemma}
We refer the reader to~\cite{genest1984} for a proof.

\begin{lemma}
\label{lem:RPC_representation}
\textbf{Representation of a pooling operator with RPC}~\citep[eq. 3.1]{genest1984}.
The only relative propensity consistent operator can always be represented by
\[ \mathcal{T} \left( \boldsymbol F_\theta \right)(\theta) = \boldsymbol B\left( \boldsymbol F_\theta \right) c(\theta) \prod_{i=0}^K \left[f_i(\theta) \right]^{w_i},\]
with $\boldsymbol B\left( \boldsymbol F_\theta \right) > 0$, $c(\theta) >0$  and $w_0, w_1, \ldots, w_K \geq 0$ arbitrary. 
\end{lemma}
Again, see~\cite{genest1984} for a proof.

\section*{The result}

Now we can state and prove the result (Remark~\ref{rmk:concavity}).
\begin{remark}
\label{rmk:concavity}
\textbf{Log-concavity}. 
If $\mathbf{F}_{\theta}$ is a set of log-concave distributions, then $\pi(\theta\mid \boldsymbol \alpha)$ is also log-concave.
Moreover, logarithmic pooling is the only pooling operator to preserve log-concavity.
\end{remark}
\begin{proof}
First, we will show by direct calculation that logarithmic pooling (LP) leads to a log-concave distribution.
Notice that each $f_i$ can be written as $ f_i(\theta) \propto e^{\nu_i(\theta)}$, where $\nu_i(\cdot)$ is a concave function.
We can then write
\begin{align*}
 \pi(\theta \mid \boldsymbol \alpha) &\propto \prod_{i=0}^{K} [\exp(\nu_i(\theta))]^{\alpha_i},\\
             &\propto \exp(\nu^{\ast}(\theta)),
\end{align*}
 where $\nu^{\ast}(\theta) = \sum_{i=0}^{K}\alpha_i\nu_i(\theta)$ is a concave function because it is a linear combination of concave functions.
 
 We will now show that LP is the only operator that guarantees log-concavity when $\boldsymbol F_\theta$ is a set of concave distributions.
 First, recall that LP is the only pooling operator that enjoys RPC (Lemma~\ref{lem:RPC_LP}).
 Then, with the goal of obtaining a contradiction, suppose that there exists a pooling operator $\mathcal{T}$ that is log-concave but does not enjoy RPC.
 From Lemma~\ref{lem:RPC_representation}, we know that $\mathcal{T}$ cannot be represented as $\boldsymbol B(\boldsymbol F_\theta) c(\theta) \prod_{i=0}^K f_i(\theta)^{w_i} $.
Every non-negative log-concave function $g(\theta)$ can be represented as
 \begin{equation}
 \label{eq:lc_rep}
  g(\theta) = a \cdot c(\theta) \cdot h(\theta),
 \end{equation}
with $a \geq 0$ and $c(\theta)$ and $h(\theta)$ non-negative and log-concave.
But under the assumptions on $\boldsymbol F_\theta$, we have that $h(\theta) := \prod_{i=0}^K f_i(\theta)^{w_i}$ is non-negative and log-concave and therefore $\mathcal{T}$ can in fact be represented in the form of~(\ref{eq:lc_rep}) and thus the form of Lemma~\ref{lem:RPC_representation}, a contradiction. 
\end{proof}

\section*{Acknowledgements}
To Professor Christian Genest for taking a look at this note.
\bibliography{../manuscript/pooling.bib}

% \begin{figure}[!ht]
% \centering
% \includegraphics[width=\textwidth, height = 15cm]{figures/}
% \caption{\textbf{}.
% }
% \label{fig:}
% \end{figure}
%%
% \begin{figure}
% \hfill
% \subfigure[Title A]{\includegraphics[width=5cm]{img1}}
% \hfill
% \subfigure[Title B]{\includegraphics[width=5cm]{img2}}
% \hfill
% \caption{\textbf{}.
% }
% \label{fig:}
% \end{figure}
\end{document}          
